\documentclass{article}
\usepackage{minted}
\newcommand\thickbar[1]{\accentset{\rule{.4em}{.8pt}}{#1}}

\title{Adversarial Models: Computer Security Issues}
\author{Auriane Blarre, Romain Kakko-Chiloff and Louis R\'emus}
\date{November 3rd,  2017}

\begin{document}
\maketitle

\section{Introduction}
Adversary attacks "trick" machine learning algorithms into making a wrong prediction by adding specific noise to the input data. There are two types of attacks:

\begin{itemize}
    \item Non targeted attacks simply aim at making the algorithm make an erroneous prediction
    \item Targeted attacks want the algorithm to make a specific prediction
\end{itemize}

More specifically, we will focus on the problem of image classification. Our goals are to:

\begin{itemize}
    \item Implement classification algorithm to tag our image set with good accuracy
    \item Find out for each algorithm a distribution of noise not discernable by the naked eye that would trick the algorithm into making a wrong prediction (targeted and non targeted)
\end{itemize}

Then we will compare how resistant the algorithms. We will also rate how easy each algorithm is to attack as a function of its parameters. Does increasing biais make an algorithm more resistant?


We plan on testing the following models:

\begin{itemize}
    \item Neural Networks
    \item SVM
    \item Ridge regression and softmax classifier
\end{itemize}

We plan on using the dataset of images provided by Google Brain for the Kaggle challenge "NIPS 2017: Adverserial Learning Developement" :
\newline
\url{https://www.kaggle.com/benhamner/adversarial-learning-challenges-getting-started}


\section{Timeline}

As the project lasts only one month, we will have to deploy an efficient methodology. We have decided to work with weekly objectives as it is mentioned in the following :
\begin{itemize}
    \item Week 1 : Data Profiling 
    \item Week 2 : Creation of a classifier that detects attacks on a simple model
    \item Week 3 : Work on different models
    \item Week 4 : Finalization of the results
\end{itemize}


\end{document}